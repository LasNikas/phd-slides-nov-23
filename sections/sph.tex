\section{Smoothed Particle Hydrodynamics (SPH)}

\begin{frame}
    \frametitle{What is SPH?}

    \begin{quote}
        \glqq{}A mesh"=free method for the discretization of functions and partial
        differential operators.\grqq{} \cite{SPH_Tutorial_2019}
    \end{quote}

    \begin{columns}[T] % align columns
        \begin{column}{.4\textwidth}
            % ==============================================================
            % ==== Linke Seite

            \vspace*{0.2cm}
            \begin{itemize}
                \item Lagrangian approach
            \end{itemize}

            \vspace*{1.6cm}

            \begin{itemize}
                \item Discretization of functions $A_i$ and fields $\bmt{A}_i$ based on a
                      kernel $W$ interpolation.
            \end{itemize}
            \vspace*{1.2cm}

            \begin{itemize}
                \item Approximation of differential operators by exact derivative of the
                      kernel function \cite{Price2012}.
            \end{itemize}
            % ==============================================================
        \end{column}
        \hfill
        \begin{column}{.6\textwidth}
            % ==============================================================
            % ==== Rechte Seite
            \begin{figure}[h]
                \centering
                \def\svgwidth{0.25\textwidth}
                \input{graphics/sph/lagragian.pdf_tex}
                \label{fig: lagragian}
            \end{figure}
            \vspace*{0.3cm}

            \begin{figure}[h]
                \centering
                \def\svgwidth{0.5\textwidth}
                \input{graphics/sph/functions_and_fields.pdf_tex}
                \label{fig: function and fields}
            \end{figure}

            \begin{align}
                A(\bmt{r})        & \approx \int A(\bmt{r}') W(\bmt{r}-\bmt{r}', h)
                \mathrm{d}\bmt{r}'                                                      \\
                \nabla A(\bmt{r}) & \approx \int A(\bmt{r}') \nabla W(\bmt{r}-\bmt{r}',
                h) \mathrm{d}\bmt{r}'
            \end{align}
            % ==============================================================
        \end{column}
    \end{columns}
\end{frame}

\begin{frame}
    \frametitle{SPH discretization}

    \begin{columns}[T] % align columns
        \begin{column}{.6\textwidth}
            % ==============================================================
            % ==== Linke Seite
            \begin{itemize}
                \item Interpolating through a convolution integral \cite{Price2012}:
            \end{itemize}
            \begin{align}
                (A * W)(\bmt{r}_i) & =
                \int \frac{A(\bmt{r}')}{\rho (\bmt{r}')} W(\bmt{r}_i-\bmt{r}', h)
                \underbrace{\rho (\bmt{r}')\mathrm{d}\bmt{r}'}_{\mathrm{d}m'}
                \\
                                   & \approx \sum_{b \in \mathcal{N}} A_b \frac{m_b}{\rho
                    (\bmt{r}_b)}
                W(\underbrace{\bmt{r}_a-\bmt{r}_b}_{\bmt{r}_{ab}}, h)
            \end{align}
            \begin{itemize}
                \item The density is computed using a weighted summation over neighboring
                      particles:
            \end{itemize}
            \begin{align}
                \rho(\bmt{r}_a) =  \sum_{b \in \mathcal{N}} m_b W(\bmt{r}_{ab}, h)
            \end{align}
            % ==============================================================
        \end{column}
        \hfill
        \begin{column}{.3\textwidth}
            % ==============================================================
            % ==== Rechte Seite
            \begin{figure}[h]
                \centering
                \def\svgwidth{\textwidth}
                \input{graphics/sph/integral_to_sum.pdf_tex}
                \label{fig: integral to sum}
            \end{figure}
            \begin{figure}[b!]
                \centering
                \def\svgwidth{0.7\textwidth}
                \input{graphics/sph/density_calculation.pdf_tex}
                \label{fig: density calculation}
            \end{figure}
            % ==============================================================
        \end{column}
    \end{columns}

\end{frame}

\setbeamertemplate{frame footer}{\cite{O_Connor_2021}}
\begin{frame}
    \frametitle{Fluid Structure Interaction (FSI)}
    Interaction of fluid particle $a$ with sets of other particles containing the
    neighboring
    fluid, boundary and structure particles:
    \begin{align*}
        \td{\ten{v}_a}{t} = & - \sum_{b \in \mathcal{N}_{\text{fluid}}}m_b
        \left(\frac{p_b}{\rho_b^2}+\frac{p_a}{\rho_a^2} +
        \underbrace{\Pi_{ab}}_{\text{viscosity}}
        \right) \nabla_a W_{ab}                                                \\
                            & - \sum_{b \in \mathcal{N}_{\text{boundary}}}m_b
        \left(\frac{p_b}{\rho_b^2}+\frac{p_a}{\rho_a^2} +
        \Pi_{ab}\right) \nabla_a W_{ab}
        \\
                            & - \sum_{b \in \mathcal{N}_{\text{structure}}}m_b
        \left(\frac{p_b}{\rho_b^2}+\frac{p_a}{\rho_a^2} +
        \Pi_{ab}\right) \nabla_a W_{ab} + \ten{g}
    \end{align*}
\end{frame}
\begin{frame}
    \frametitle{Fluid Structure Interaction (FSI)}
    Interaction of structure particle $a$ with sets of other particles containing the
    neighboring
    fluid and structure particles:
    \begin{align*}
        \td{\ten{v}_a}{t} = & - \frac{m_a}{m_{0a}}\sum_{b \in
        \mathcal{N}_{\text{fluid}}}m_b
        \left(\frac{p_b}{\rho_b^2}+\frac{p_a}{\rho_a^2} +
        \Pi_{ab}
        \right) \nabla_a W_{ab}
        \\
                            & +\sum_{b\in \mathcal{N}_{\text{structure}}} m_{0b}
        \left( \frac{\ten{P}_a \ten{L}_{0a}}{\rho_{0a}^2} + \frac{\ten{P}_b
            \ten{L}_{0b}}{\rho_{0b}^2} \right)
        \nabla_{0a} W(\ten{X}_{ab}) +
        \underbrace{\frac{\ten{f}_a^{PF}}{m_{0a}}}_{\text{optional penalty force}} + \:
        \ten{g}
    \end{align*}
    where $m_a$ and $\rho_a$ refer to the hydrodynamic mass and density of a structure
    particle. Whereas $m_{0a}$ and $\rho_{0a}$ refer to the material mass and density of
    structure particle $a$. And $\ten{P}$ is the first Piola-Kirchhoff (PK1) stress tensor.
\end{frame}
\setbeamertemplate{frame footer}{}

\begin{frame}
    \frametitle{Problem - Pressure field of structure and boundary}
    \texttt{TrixiParticles.jl} provides five options to compute the hydrodynamic boundary (structure) density and
    pressure. The one which usually yields the best results is described in
    \cite{Adami_2012}. The pressure of a boundary (structure) particle is obtained by
    extrapolating the pressure of the fluid:
    \begin{align}
        p_{\text{solid}} = \frac{\sum_\text{fluid} (p_\text{fluid} + \rho_\text{fluid} (\bm{g} - \bm{a}_{\text{solid}}) \cdot \ten{x}_{\text{solid-fluid}}) W(
            \ten{x}_{\text{solid-fluid}} )}{\sum_\text{fluid} W( \ten{x}_{\text{solid-fluid}})},
    \end{align}
    The hydrodynamic density for boundary and structure is then obtained by the inverse equation of state.

\end{frame}
