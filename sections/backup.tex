\section{Backup}
\begin{frame}
    \frametitle{SPH discretization (fluid)}
    \begin{columns}[T] % align columns
        \begin{column}{.4\textwidth}
            % ==============================================================
            % ==== Linke Seite
            The governing equations in a Lagrangian framework read as
            \begin{align}
                \label{euler dv}
                \td{\ten{v}}{t} & = - \frac{1}{\rho} \nabla p + \frac{1}{\rho} \ten{F}^{(\nu)}
                +
                \ten{g},
                \\
                \label{euler drho}
                \td{\rho}{t}    & = -\rho \nabla \cdot \ten{v},
                \\
                \td{\ten{r}}{t} & = \ten{v},
                \\
                p               & = F(\rho) \: .
            \end{align}
            % ==============================================================
        \end{column}
        \hfill
        \begin{column}{.5\textwidth}
            % ==============================================================
            % ==== Rechte Seite
            Using Monaghans \cite{Monaghan2005} SPH interpolation rules, eq. (\ref{euler dv})
            and (\ref{euler drho}) become
            \begin{align}
                \td{\ten{v}_a}{t} & = - \sum_{b \in \mathcal{N}}m_b
                \left(\frac{p_b}{\rho_b^2}+\frac{p_a}{\rho_a^2} +
                \underbrace{\Pi_{ab}}_{\text{viscosity}}
                \right) \nabla_a W_{ab} + \ten{g},
                \\
                \label{sph dv}
                \td{\rho_a}{t}    & = \sum_{b \in \mathcal{N}} m_b (v_a - v_b) \cdot \nabla_a
                W_{ab}	\: .
            \end{align}
            % ==============================================================
        \end{column}
    \end{columns}
    \vspace{1cm}
    The equation of state to describe the relationship between pressure and density of water up
    to high pressures is given as
    \begin{align}
        p =\frac{\rho_0 c^2}{\gamma} \left(\left(\frac{\rho}{\rho_0}\right)^\gamma - 1\right) +
        p_0 - p_{\text{background}}
    \end{align}
\end{frame}

\begin{frame}
    \frametitle{SPH discretization (solid)}

    The conservation equation of linear momentum is given by
    \begin{align}
        \frac{\mathrm{d} \ten{v}}{\mathrm{d}t} & = \frac{1}{\rho} \nabla_0 \ten{P}\T +
        \ten{g} \: ,
        \label{conservation momentum}
    \end{align}
    where $\ten{P}$ is the first Piola-Kirchhoff (PK1) stress tensor. \\
    \vspace*{0.5cm}
    The discretized version of this equation is given by O'Connor et al
    \cite{O_Connor_2021}:
    \begin{align}
        \td{\ten{v}_a}{t} = \sum_{b\in \mathcal{N}} m_{0b}
        \left( \frac{\ten{P}_a \ten{L}_{0a}}{\rho_{0a}^2} + \frac{\ten{P}_b
            \ten{L}_{0b}}{\rho_{0b}^2} \right)
        \nabla_{0a} W(\ten{X}_{ab}) +
        \underbrace{\frac{\ten{f}_a^{PF}}{m_{0a}}}_{\text{optional penalty force}} + \:
        \ten{g},
    \end{align}
    %penalty force: Korrigierende Kraft, um hourglass modes (FEM: Unterintegrierten Elementen) zu verhindern.
    with the inverse correction (renormalisation) matrix
    % to handle an incomplete kernel support at the edges of the domain.
    \begin{align}
        \ten{L}_{0a} := \left( \sum_{b \in \mathcal{N}} \frac{m_{0b}}{\rho_{0b}}
        \nabla_{0a}
        W(\ten{X}_{ab}) \ten{X}_{ab}^T \right)^{-1} \in \R^{d \times d}.
    \end{align}
    The difference in the initial (reference) configuration is denoted as $\ten{X}_{ab}
        = \ten{X}_a - \ten{X}_b$ .
\end{frame}

\begin{frame}
    \frametitle{PK1 stress tensor}
    For the computation of the PK1 stress tensor, the deformation gradient $\ten{J}$ is
    computed per particle as
    \begin{align}
        \ten{J}_a = \sum_b \frac{m_{0b}}{\rho_{0b}} \ten{x}_{ba} (\ten{L}_{0a}\nabla_{0a}
        W(\ten{X}_{ab}))^T
    \end{align}
    with $1 \leq i,j \leq d$.
    From the deformation gradient, the Green-Lagrange strain
    \begin{align}
        \ten{E} = \frac{1}{2}(\ten{J}^T\ten{J} - \ten{I})
    \end{align}
    and the second Piola-Kirchhoff stress tensor
    \begin{align}
        \ten{S} = \lambda \operatorname{tr}(\ten{E}) \ten{I} + 2\mu \ten{E}
    \end{align}
    are computed to obtain the PK1 stress tensor as $\ten{P} = \ten{J}\ten{S}$.
    Here,
    \begin{align}
        \mu = \frac{E}{2(1 + \nu)}\qquad \text{and} \qquad \lambda = \frac{E\nu}{(1 +
            \nu)(1 -
            2\nu)}
    \end{align}
    are the Lamé coefficients, where $E$ is the Young's modulus and $\nu$ is the Poisson
    ratio.

\end{frame}
